\documentclass[letterpaper]{article}

\usepackage{fancyhdr}
\usepackage{extramarks}
\usepackage{amsmath}
\usepackage{amsthm}
\usepackage{amsfonts}
\usepackage{tikz}
\usepackage[plain]{algorithm}
\usepackage{algpseudocode}
\usepackage{listings}

\usetikzlibrary{automata,positioning}

%
% Basic Document Settings
%

\topmargin=-0.45in
\evensidemargin=0in
\oddsidemargin=0in
\textwidth=6.5in
\textheight=9.0in
\headsep=0.25in

\linespread{1.1}

\pagestyle{fancy}
\lhead{\hmwkAuthorName}
\chead{\hmwkClass\ (\hmwkClassInstructor): \hmwkTitle}
\rhead{\firstxmark}
\lfoot{\lastxmark}
\cfoot{\thepage}

\renewcommand\headrulewidth{0.4pt}
\renewcommand\footrulewidth{0.4pt}

\setlength\parindent{0pt}

%
% Create Problem Sections
%

\newcommand{\enterProblemHeader}[1]{
    \nobreak\extramarks{}{Problem \arabic{#1} continued on next page\ldots}\nobreak{}
    \nobreak\extramarks{Problem \arabic{#1} (continued)}{Problem \arabic{#1} continued on next page\ldots}\nobreak{}
}

\newcommand{\exitProblemHeader}[1]{
    \nobreak\extramarks{Problem \arabic{#1} (continued)}{Problem \arabic{#1} continued on next page\ldots}\nobreak{}
    \stepcounter{#1}
    \nobreak\extramarks{Problem \arabic{#1}}{}\nobreak{}
}

\setcounter{secnumdepth}{0}
\newcounter{partCounter}
\newcounter{homeworkProblemCounter}
\setcounter{homeworkProblemCounter}{1}
\nobreak\extramarks{Problem \arabic{homeworkProblemCounter}}{}\nobreak{}

%
% Homework Problem Environment
%
% This environment takes an optional argument. When given, it will adjust the
% problem counter. This is useful for when the problems given for your
% assignment aren't sequential. See the last 3 problems of this template for an
% example.
%
\newenvironment{homeworkProblem}[1][-1]{
    \ifnum#1>0
        \setcounter{homeworkProblemCounter}{#1}
    \fi
    \section{Problem \arabic{homeworkProblemCounter}}
    \setcounter{partCounter}{1}
    \enterProblemHeader{homeworkProblemCounter}
}{
    \exitProblemHeader{homeworkProblemCounter}
}

%
% Homework Details
%   - Title
%   - Due date
%   - Class
%   - Section/Time
%   - Instructor
%   - Author
%

\newcommand{\hmwkTitle}{Homework\ \#1}
\newcommand{\hmwkDueDate}{September 3, 2016}
\newcommand{\hmwkClass}{CS5343.001}
\newcommand{\hmwkClassTime}{001}
\newcommand{\hmwkClassInstructor}{Professor Greg Ozbirn}
\newcommand{\hmwkAuthorName}{Lizhong Zhang}

%
% Title Page
%

\title{
    \vspace{2in}
    \textmd{\textbf{\hmwkClass:\ \hmwkTitle}}\\
    \normalsize\vspace{0.1in}\small{Due\ on\ \hmwkDueDate\ at 11:59pm}\\
    \vspace{0.1in}\large{\textit{\hmwkClassInstructor}}
    \vspace{3in}
}

\author{\textbf{\hmwkAuthorName}}
\date{}

\renewcommand{\part}[1]{\textbf{\large Part \Alph{partCounter}}\stepcounter{partCounter}\\}

%
% Various Helper Commands
%

% Useful for algorithms
\newcommand{\alg}[1]{\textsc{\bfseries \footnotesize #1}}

% For derivatives
\newcommand{\deriv}[1]{\frac{\mathrm{d}}{\mathrm{d}x} (#1)}

% For partial derivatives
\newcommand{\pderiv}[2]{\frac{\partial}{\partial #1} (#2)}

% Integral dx
\newcommand{\dx}{\mathrm{d}x}

% Alias for the Solution section header
\newcommand{\solution}{\vspace{0.1in}\textbf{\large Solution}}

% Probability commands: Expectation, Variance, Covariance, Bias
\newcommand{\E}{\mathrm{E}}
\newcommand{\Var}{\mathrm{Var}}
\newcommand{\Cov}{\mathrm{Cov}}
\newcommand{\Bias}{\mathrm{Bias}}


\lstloadlanguages{Java}
%\lstset{                                            % general command to set parameter(s)
%    basicstyle=\footnotesize,                         % print whole listing small
    %keywordstyle=                                   % underlined bold black keywords
    %identifierstyle=,                               % nothing happens
    %commentstyle=\color{white},                     % white comments
    %stringstyle=,                                   % typewriter type for strings
%    numbers=left,                    % where to put the line-numbers; possible values are (none, left, right)
%    numbersep=5pt,                   % how far the line-numbers are from the code
%    keywordstyle=\color{blue},       % keyword style
%    showstringspaces=false                          % no special string spaces
%}

\lstdefinestyle{customJava}{
    belowcaptionskip=1\baselineskip,
    breaklines=true,
    frame=single,
    xleftmargin=10pt,
    language=Java,
    showstringspaces=false,
    numbers=left,
    numberstyle=\footnotesize\ttfamily,
    stepnumber=1,
    numbersep=5pt,
    basicstyle=\footnotesize\ttfamily,
    keywordstyle=\bfseries\color{blue!40!black},
    commentstyle=\itshape\color{purple!40!black},
    %identifierstyle=\color{blue},
    stringstyle=\color{orange},
}

\begin{document}

\maketitle

\pagebreak

\begin{homeworkProblem}
    Suppose your calculator only did base 10 logarithms. Write an expression to compute log base 2 of 2048 using only log base 10.

    \solution

    \[\log_{2} 2048 = \frac{\log_{10} 2048}{\log_{10} 2} = 11\]

\end{homeworkProblem}

\begin{homeworkProblem}
    Express the following summation in closed form (an expression that can be directly computed from k).
    
    (Refer to slide 11)

    \[3 + 5 + 7 + 9 + ... + 2k+1\]

    \solution

    \[\sum _{i=1}^k {(2i + 1)} = \frac{(3+2k+1)k}{2} = k(k+2)\] 


\end{homeworkProblem}

\begin{homeworkProblem}

    Proof by counterexample.

    \begin{quote}
    Prove that the following statement is false:   $n^3 > 2^n$  for any $n >= 1$.
    \end{quote}

    \textbf{Justification}
    \\

    Suppose $n=1$, $n^3=1$ and $2^n=2$. Apparently, $n^3 < 2^n$.

    Thus, the statement is false.

\end{homeworkProblem}

\begin{homeworkProblem}

    Proof by contradiction.

    \begin{quote}
    Prove that the following statement is true: the square of an even number is also even.
    \end{quote}

    \textbf{Justification}
    \\

    Assume the statement is false: give an even number $x=2k$, then $x^2 = (2k)^2 = 4k^2$ is an odd number. This indicate that an even number equals to an odd number, which is impossible.

    Thus, the statement is true. The square of an even number is also even.

\end{homeworkProblem}

\begin{homeworkProblem}

    Induction proofs.
    \\

    \textbf{Part One}

    Prove by induction:

    \[\sum _{i=1}^n {i^3} =\frac{(n^2)(n+1)^2}{4}\]

    \textbf{Justification}
    \\

    \emph{Base case}: $n=1$.

    \[\sum _{i=1}^1 {i^3} =\frac{(1^2)(1+1)^2}{4} = 1\]

    So, it is true for $n=1$.

    \emph{Induction step}: Assume it is true for k, that $\sum _{i=1}^k {i^3} =\frac{(k^2)(k+1)^2}{4}$.

    Show true for $k + 1$;
    \[
        \begin{split}
            \sum _{i=1}^{k+1} {i^3} &= \sum _{i=1}^k {i^3} + (k+1)^3
            \\
            &= \frac{k^2(k+1)^2}{4} + (k+1)^3
            \\
            &= \frac{k^2(k+1)^2 + 4(k+1)^3}{4}
            \\
            &= \frac{k^2(k+1)^2 + 4(k+1)^2(k+1)}{4}
            \\
            &= \frac{(k+1)^2(k^2+4k+4)}{4}
            \\
            &= \frac{(k+1)^2(k+2)^2}{4}
        \end{split}
    \]
    \\

    \emph{Conclusion}: by induction, the statement holds true for all $n$.
    \\

    \textbf{Part Two}

    Prove by induction:

    \begin{quote}
        $n^2 - n$ is even for any $n >= 1$.
    \end{quote}


    \textbf{Justification}
    
    \emph{Base case}: $n=1$, then, $n^2 - n = 1^2 -1 = 0$ is even. So, it is true for $n=1$.

    \emph{Induction step}: Assume it is true for k, that $k^2 - k$ is even, so $k^2 - k = 2a$.

    Show true for $k + 1$;
    \[
        \begin{split}
           (k+1)^2 - (k+1) &= (k+1)((k+1)-1)
            \\
            &= (k + 1) k
            \\
            &= k^2 + k
            \\
            &= k^2-k+2k
            \\
            &= 2a+2k
        \end{split}
    \]

    We know $2a$ is even and $2k$ is even for all $n >= 1$.

    So, $2a + 2k$ is even.

    \emph{Conclusion}: by induction, the statement holds true for all $n >= 1$.




\end{homeworkProblem}

\begin{homeworkProblem}

    Recursion

    \begin{footnotesize}
    {Note: You can use Java or pseudocode for these. If pseudocode then the logic must be complete and easy to understand.}
    \end{footnotesize}


    \textbf{Part One}

    Write a recursive function that when passed a value n displays.

    \begin{quote}
        $n$  $(n-1)$  $(n-2)$  $(n-3)$  ...  $0$ ... $(n-3)$ $(n-2)$ $(n-1)$ $n$
    \end{quote}

    for example, if passed 5 displays:

    \begin{quote}
        5 4 3 2 1 0 1 2 3 4 5
    \end{quote}

    \solution

    The Java code is as follow:

    \lstinputlisting[style=customJava]{Recursion1.java}

    \textbf{Part Two}

    Write a recursive function that receives an array of integers
        and a position as parameters and returns the count of odd
        numbers in the array.  Let each recursive call consider the
        next integer in the array.


    \solution

    The Java code is as follow:
    
    \lstinputlisting[style=customJava]{Recursion2.java}

\end{homeworkProblem}

\begin{homeworkProblem}
	
   {Suppose there exists a generic Java class named Pair with type
   parameter T that stores two objects with get and set methods
   for each.  Write the statements necessary to create an object
   of type Pair with String as its type parameter, and use the set
   methods to set the two strings, then the get methods to retrieve
   them for printing.  Note that you do not need to write the 
   Pair class itself.}


    \solution

    The Java code is as follow:
    
    \lstinputlisting[style=customJava]{Pair.java}

\end{homeworkProblem}

\end{document}